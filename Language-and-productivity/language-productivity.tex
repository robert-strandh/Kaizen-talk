\documentclass{slides}
\usepackage[utf8]{inputenc}
\usepackage{graphics}
\usepackage{portland}
\usepackage{epsfig}
\usepackage{alltt}
\usepackage{moreverb}
\usepackage{url}
\usepackage[dvips,usenames]{color}

\definecolor{MyLightMagenta}{rgb}{1,0.7,1}
\definecolor{darkgreen}{rgb}{0.1,0.7,0.1}

\newcommand{\darkgreen}[1]{\textcolor{darkgreen}{#1}}
\newcommand{\red}[1]{\textcolor{red}{#1}}
\newcommand{\thistle}[1]{\textcolor{Thistle}{#1}}
\newcommand{\apricot}[1]{\textcolor{Apricot}{#1}}
\newcommand{\melon}[1]{\textcolor{Melon}{#1}}
\newcommand{\dandelion}[1]{\textcolor{Dandelion}{#1}}
\newcommand{\green}[1]{\textcolor{OliveGreen}{#1}}
\newcommand{\lavender}[1]{\textcolor{Lavender}{#1}}
\newcommand{\mylightmagenta}[1]{\textcolor{MyLightMagenta}{#1}}
\newcommand{\blue}[1]{\textcolor{RoyalBlue}{#1}}
\newcommand{\darkorchid}[1]{\textcolor{DarkOrchid}{#1}}
\newcommand{\orchid}[1]{\textcolor{Orchid}{#1}}
\newcommand{\brickred}[1]{\textcolor{BrickRed}{#1}}
\newcommand{\peach}[1]{\textcolor{Peach}{#1}}
\newcommand{\bittersweet}[1]{\textcolor{Bittersweet}{#1}}
\newcommand{\salmon}[1]{\textcolor{Salmon}{#1}}
\newcommand{\yelloworange}[1]{\textcolor{YellowOrange}{#1}}
\newcommand{\periwinkle}[1]{\textcolor{Periwinkle}{#1}}

\newcommand{\names}[1]{\periwinkle{#1}}
\newcommand{\motcle}[1]{\mylightmagenta{#1}}
\newcommand{\classname}[1]{\darkgreen{#1}}
\newcommand{\str}[1]{\yelloworange{#1}}
\newcommand{\defun}[1]{\orchid{#1}}
\newcommand{\ti}[1]{\begin{center}\Large{\textcolor{blue}{#1}}\end{center}}
\newcommand{\alert}[1]{\thistle{#1}}
\newcommand{\lispprint}[1]{\dandelion{#1}}
\newcommand{\lispvalue}[1]{\red{#1}}
\newcommand{\tr}[1]{\texttt{\red{#1}}}
\newcommand{\emc}[1]{\red{#1}}
\newcommand{\lispobj}[1]{\green{\texttt{#1}}}
\def\prompt{{\textcolor{Orchid}{CL-USER>}}}
\newcommand{\promptp}[1]{\textcolor{Orchid}{#1>}}

\newcommand{\Comment}[1]{
\begin{center}
\textcolor{yellow}
{#1}
\end{center}
}

\def\bs{$\backslash$}
\def\inputfig#1{\input #1}
\def\inputtex#1{\input #1}

\begin{document}
\landscape
\setlength{\oddsidemargin}{1cm}
\setlength{\evensidemargin}{1cm}
\setlength{\marginparwidth}{1cm}
\setlength{\parskip}{0.5cm}
\setlength{\parindent}{0cm}
%-----------------------------------------------------------
\begin{slide}\ti{Programming language productivity}
\vskip 0.5cm
\begin{center}
Robert Strandh \\
Université de Bordeaux \\
Bordeaux, France
\end{center}
\vfill\end{slide}
%-----------------------------------------------------------
\begin{slide}\ti{Overview of talk}

  \begin{itemize}
  \item Programmer productivity.
  \item Presentation of paper by Hudak and Jones.
  \item Interpretation of the results in the paper.
  \item How to choose a language.
  \end{itemize}

\vfill\end{slide}
%-----------------------------------------------------------
\begin{slide}\ti{Programmer productivity}

Not well defined, so can not be measured.

Main problem: measuring the extent of the task to accomplish

Using ``lines of code'' as a measure has been likened to measuring the
performance of an automobile by its weight.

The idea behind \emph{function points} is better, but the definition
of that concept is no longer relevant.  It is (and was) also too
restrictive.

\vfill\end{slide}
%-----------------------------------------------------------
\begin{slide}\ti{Comparing productivity}

If we fix the task, we can compare productivity as a function of other
parameters such as tools, training, and development method.

\vfill\end{slide}
%-----------------------------------------------------------
\begin{slide}\ti{Hudak and Jones}

Creators of the programming language Haskell.

Haskell is a purely functional programming language that uses
\emph{lazy evaluation}.

\vfill\end{slide}
%-----------------------------------------------------------
\begin{slide}\ti{Hudak and Jones}

Functional programming turned out to be hard to ``sell''.

In 1994, \emph{rapid prototyping} became a hot topic.

Rapid prototyping evolved from the statement by Frederick P. Brooks Jr
(author of ``The Mythical man-month''): ``Plan to throw one way.  You
will anyway.''

So the Haskell creators attempted to promote the language as one
adapted for this task.

What happened to rapid prototyping?

\vfill\end{slide}
%-----------------------------------------------------------
\begin{slide}\ti{Message of this talk}

THE INFLUENCE OF THE PROGRAMMING LANGUAGE ON PROGRAMMER PRODUCTIVITY
MIGHT VERY WELL BE HUGE.  TAKING THIS INFLUENCE INTO ACCOUNT MIGHT
MAKE YOUR PROGRAMMING PROJECT SIGNIFICANTLY LESS EXPENSIVE TO
ACCOMPLISH.

\vfill\end{slide}
%-----------------------------------------------------------
\begin{slide}\ti{US Navy conducted an experiment}

In 1993, The US Navy wanted to evaluate different languages for rapid
prototyping.

A sub-system of a warfare system was chosen, namely a \emph{geometric
  region server} (or geo-server).

An informal description of the problem was given to all participants.
\vfill\end{slide}
%-----------------------------------------------------------
\begin{slide}\ti{Experts participated}

``The participants, each considered an expert programmer in one of the
  programming languages being tested, was asked to write a fully
  functional prototype of the geo-server, while keeping track of
  software development metrics such as development time and lines of
  code and documentation.''

\vfill\end{slide}
%-----------------------------------------------------------
\begin{slide}\ti{Prototypes and metrics collected}

``The prototypes and metrics were collected by [the navy] and
  distributed to a distinguished panel of computer scientists and
  software engineers for evaluation along several dimensions.''

\vfill\end{slide}
%-----------------------------------------------------------
\begin{slide}\ti{Experiment has many flaws}

  \begin{itemize}
  \item The problem is too small.
  \item Specification ambiguous and imprecise (which is normal).
  \item Participants were trusted to report their metrics.
  \item No code was run by the panel.
  \end{itemize}

\vfill\end{slide}
%-----------------------------------------------------------
\begin{slide}\ti{Advantages of experiment}

  \begin{itemize}
  \item Experiment conducted by people without interest in a
    particular language, and who are ``highly regarded, with many
    years experience developing large, complex, software systems.''
  \item The programming task, although simplified, is real.
  \item Each program written by an expert in the language used.
  \end{itemize}

\vfill\end{slide}
%-----------------------------------------------------------
\begin{slide}\ti{Specification of the task}

  \begin{itemize}
  \item The input consists of positions of ships, airplanes, and other
    objects.
  \item Output consists of relationships between those objects.
  \item Each object defines a \emph{zone} of interest, according to
    the type of the object, and whether the object is friendly or
    hostile.
  \end{itemize}

Details are not important for this talk.

\vfill\end{slide}
%-----------------------------------------------------------
\begin{slide}\ti{Essence of specification}

Write a program that can manage two-dimensional \emph{zones} of
different shapes (discs, bands, cones).

Zones are combined using operations such as \emph{union},
\emph{intersection}, and \emph{complement}.

\vfill\end{slide}
%-----------------------------------------------------------
\begin{slide}\ti{Result}

  \begin{tabular}{|l|r|r|r|}
    \hline
    Language & Lines of & Lines of       & Development\\
             & code     & documentation  & time (hours)\\
    \hline
    \hline
    (1) Haskell     &   85 & 465 & 10\\
    \hline
    (2) Ada         &  767 & 714 & 23\\
    \hline
    (3) Ada 9X      &  800 & 200 & 28\\
    \hline
    (4) C++         & 1105 & 130 & -\\
    \hline
    (5) Awk/Nawk    &  250 & 150 & -\\
    \hline
    (6) Rapide      &  157 &   0 & 54\\
    \hline
    (7) Griffin     &  251 &   0 & 34\\
    \hline
    (8) Proteus     &  293 &  79 & 26\\
    \hline
    (9) Relational Lisp & 274 & 12 & 3\\
    \hline
    (10) Haskell    &  156 & 112 &  8\\
    \hline
  \end{tabular}

\vfill\end{slide}
%-----------------------------------------------------------
\begin{slide}\ti{Interpretation of result}

  \begin{itemize}
  \item The result suggests a \emph{factor 20} difference in code size
    and development time is possible.
  \item Clearly, such huge difference is not likely for a large
    project, though.
  \item The results also show that prior knowledge of a language is
    not as important as one might think.
  \end{itemize}

\vfill\end{slide}
%-----------------------------------------------------------
\begin{slide}\ti{Further interpretation}

Even though a factor 20 (i.e., 2000\%) may not be possible, consider
the possibility that there is \emph{some} difference, say for instance
20\%.

Such a difference may mean that a task could be accomplished in 4
years rather than 5.  Several months could then be used to investigate
other languages, train the staff, hire new programmers, etc.

\vfill\end{slide}
%-----------------------------------------------------------
\begin{slide}\ti{So how do we choose a langauge?}

\vfill\end{slide}
%-----------------------------------------------------------
\begin{slide}\ti{So how do we choose a langauge?}

It requires good knowledge of:

\begin{itemize}
\item The characteristics of several programming languages,
\item The nature of the task to be accomplished, and
\item A separate and detailed budget for each ``reasonable'' language
  choice.
\end{itemize}

The first item is the subject of a different talk.

We will look at the last item a bit more.

\vfill\end{slide}
%-----------------------------------------------------------
\begin{slide}\ti{How not to choose}

  \begin{itemize}
  \item ``We need all the speed we can get, and it is known that the
    C++ compiler generates very fast code.  Therefore we choose C++.''
  \item ``All our programmers already know Java.  Therefore we choose
    Java.''
  \item ``We have made a huge investment in programming tools for
    C\#.  Therefore we choose C\#.''
  \end{itemize}

\vfill\end{slide}
%-----------------------------------------------------------
\begin{slide}\ti{What to include in the budgets}

  \begin{itemize}
  \item Cost of acquisition of language tools.
  \item Estimated development and maintenance cost.
  \item Cost of training and hiring new staff.
  \item ...
  \end{itemize}

\vfill\end{slide}
%-----------------------------------------------------------
\begin{slide}\ti{What to include in the budgets}

  \begin{itemize}
  \item Cost of acquisition of language tools.
  \item Estimated development and maintenance cost.
  \item Cost of training and hiring new staff.
  \item ...
  \item Risk analysis.
  \end{itemize}

\vfill\end{slide}
%-----------------------------------------------------------
\begin{slide}\ti{Risk analysis}

For every major possible choice (tools, staff, development method,
etc.), make a list of possible events that might have a negative
impact on the project.

For each event, state:

\begin{itemize}
\item its likelihood,
\item the cost to the project if nothing is done,
\item actions to avoid the negative impact, and
\item the cost of those actions.
\end{itemize}

\vfill\end{slide}
%-----------------------------------------------------------
\begin{slide}\ti{Risk analysis}

Example:

Choice: Make Joe a member of the staff.  Joe is a reckless driver.

Event: Joe has a traffic accident and can no longer work on the
project.

Likelihood: Unlikely

Cost if nothing is done: The project will be delayed by six months.

Action: Hire a replacement for Joe.

Cost of action: Salary, training, etc.

\vfill\end{slide}
%-----------------------------------------------------------
\begin{slide}\ti{Risk analysis}

Example:

Choice: Use the language C\#.

Event: Microsoft is bought by Apple (or Google) and C\# is no longer
supported.

Likelihood: Unlikely

Cost if nothing is done: All code must be rewritten in Java.

Action: Obtain (buy, develop) a replacement for Microsoft C\#.

Cost of action: Cost of purchase or development.

\vfill\end{slide}
%-----------------------------------------------------------
\begin{slide}\ti{Solution}

{\tiny\begin{verbatim}
(defun inside? (point zone)
  (funcall zone point))

(defun disc (radius)
  (lambda (point) (<= (abs point) radius)))

(defun cone (angle)
  (lambda (point) (<= 0 (phase point) angle)))

(defun band (width)
  (lambda (point) (<= (abs (realpart point)) (/ width 2))))

(defun complement (zone)
  (lambda (point) (not (inside? point zone))))

(defun union (zone1 zone2)
  (lambda (point) (or (inside? point zone1) (inside? point zone2))))

(defun intersection (zone1 zone2)
  (lambda (point) (and (inside? point zone1) (inside? point zone2))))

(defun translate (zone delta)
  (lambda (point) (inside? (- point delta) zone)))

(defun rotate (zone angle)
  (lambda (point) (inside? zone (/ point (expt #c(0.0 1.0) pi)))))
\end{verbatim}
}

\vfill\end{slide}
%-----------------------------------------------------------
%% \begin{slide}\ti{}

%% \vfill\end{slide}
%% %-----------------------------------------------------------
%% \begin{slide}\ti{}

%% \vfill\end{slide}
%--------------------------------

\end{document}
 
