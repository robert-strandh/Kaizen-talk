\documentclass{slides}
\usepackage[utf8]{inputenc}
\usepackage{graphics}
\usepackage{portland}
\usepackage{epsfig}
\usepackage{alltt}
\usepackage{moreverb}
\usepackage{url}
\usepackage[dvips,usenames]{color}

\definecolor{MyLightMagenta}{rgb}{1,0.7,1}
\definecolor{darkgreen}{rgb}{0.1,0.7,0.1}

\newcommand{\darkgreen}[1]{\textcolor{darkgreen}{#1}}
\newcommand{\red}[1]{\textcolor{red}{#1}}
\newcommand{\thistle}[1]{\textcolor{Thistle}{#1}}
\newcommand{\apricot}[1]{\textcolor{Apricot}{#1}}
\newcommand{\melon}[1]{\textcolor{Melon}{#1}}
\newcommand{\dandelion}[1]{\textcolor{Dandelion}{#1}}
\newcommand{\green}[1]{\textcolor{OliveGreen}{#1}}
\newcommand{\lavender}[1]{\textcolor{Lavender}{#1}}
\newcommand{\mylightmagenta}[1]{\textcolor{MyLightMagenta}{#1}}
\newcommand{\blue}[1]{\textcolor{RoyalBlue}{#1}}
\newcommand{\darkorchid}[1]{\textcolor{DarkOrchid}{#1}}
\newcommand{\orchid}[1]{\textcolor{Orchid}{#1}}
\newcommand{\brickred}[1]{\textcolor{BrickRed}{#1}}
\newcommand{\peach}[1]{\textcolor{Peach}{#1}}
\newcommand{\bittersweet}[1]{\textcolor{Bittersweet}{#1}}
\newcommand{\salmon}[1]{\textcolor{Salmon}{#1}}
\newcommand{\yelloworange}[1]{\textcolor{YellowOrange}{#1}}
\newcommand{\periwinkle}[1]{\textcolor{Periwinkle}{#1}}

\newcommand{\names}[1]{\periwinkle{#1}}
\newcommand{\motcle}[1]{\mylightmagenta{#1}}
\newcommand{\classname}[1]{\darkgreen{#1}}
\newcommand{\str}[1]{\yelloworange{#1}}
\newcommand{\defun}[1]{\orchid{#1}}
\newcommand{\ti}[1]{\begin{center}\Large{\textcolor{blue}{#1}}\end{center}}
\newcommand{\alert}[1]{\thistle{#1}}
\newcommand{\lispprint}[1]{\dandelion{#1}}
\newcommand{\lispvalue}[1]{\red{#1}}
\newcommand{\tr}[1]{\texttt{\red{#1}}}
\newcommand{\emc}[1]{\red{#1}}
\newcommand{\lispobj}[1]{\green{\texttt{#1}}}
\def\prompt{{\textcolor{Orchid}{CL-USER>}}}
\newcommand{\promptp}[1]{\textcolor{Orchid}{#1>}}

\newcommand{\Comment}[1]{
\begin{center}
\textcolor{yellow}
{#1}
\end{center}
}

\def\bs{$\backslash$}
\def\inputfig#1{\input #1}
\def\inputtex#1{\input #1}

\begin{document}
\landscape
\setlength{\oddsidemargin}{1cm}
\setlength{\evensidemargin}{1cm}
\setlength{\marginparwidth}{1cm}
\setlength{\parskip}{0.5cm}
\setlength{\parindent}{0cm}
%-----------------------------------------------------------
\begin{slide}\ti{Choosing a programming language}
\vskip 0.5cm
\begin{center}
Robert Strandh \\
Université de Bordeaux \\
Bordeaux, France
\end{center}
\vfill\end{slide}
%-----------------------------------------------------------
\begin{slide}\ti{Overview of talk}

  \begin{itemize}
  \item Risk analysis.
  \end{itemize}

\vfill\end{slide}
%-----------------------------------------------------------
\begin{slide}\ti{Risk analysis}

For every major possible choice (tools, staff, development method,
etc.), make a list of possible events that might have a negative
impact on the project.

For each event, state:

\begin{itemize}
\item its likelihood,
\item the cost to the project if nothing is done,
\item actions to avoid the negative impact, and
\item the cost of those actions.
\end{itemize}

\vfill\end{slide}
%-----------------------------------------------------------
\begin{slide}\ti{Risk analysis}

Example:

Choice: Make Joe a member of the staff.  Joe is a reckless driver.

Event: Joe has a traffic accident and can no longer work on the
project. 

Likelihood: Unlikely

Cost if nothing is done: The project will be delayed by six months. 

Action: Hire a replacement for Joe.

Cost of action: Salary, training, etc.

\vfill\end{slide}
%-----------------------------------------------------------
\begin{slide}\ti{Risk analysis}

Example:

Choice: Use the language C\#.

Event: Microsoft is bought by Apple (or Google) and C\# is no longer
supported.

Likelihood: Unlikely

Cost if nothing is done: All code must be rewritten in Java.

Action: Obtain (buy, develop) a replacement for Microsoft C\#.

Cost of action: Cost of purchase or development.

\vfill\end{slide}
%-----------------------------------------------------------
\begin{slide}\ti{Language vs implementation}

Language: A description of the syntax and semantics of conforming
programs, and of consequences of using non-conforming constructs.

Example of the latter: In C, obtaining a pointer beyond an array has
undefined consequences.  (It is interesting to contemplate why)

\vfill\end{slide}
%-----------------------------------------------------------
\begin{slide}\ti{Language vs implementation}

Implementation: Software that accepts conforming programs and executes
them according to the language semantics, and that reports
non-conforming constructs where required by the language definition.

\vfill\end{slide}
%-----------------------------------------------------------
\begin{slide}\ti{Language vs implementation}

The distinction language/implementation is sometimes blurred:

\begin{itemize}
\item Some languages are defined by a \emph{reference
  implementation}.  Examples?
\item Some language definitions are controlled by the same
  organization that supplies some dominating implementation.
  Examples?
\end{itemize}

\vfill\end{slide}
%-----------------------------------------------------------
%% \begin{slide}\ti{}

%% \vfill\end{slide}
%% %-----------------------------------------------------------
%% \begin{slide}\ti{}

%% \vfill\end{slide}
%--------------------------------

\end{document}
 
