\documentclass{slides}
\usepackage[utf8]{inputenc}
\usepackage{graphics}
\usepackage{portland}
\usepackage{epsfig}
\usepackage{alltt}
\usepackage{moreverb}
\usepackage{url}
\usepackage[dvips,usenames]{color}

\definecolor{MyLightMagenta}{rgb}{1,0.7,1}
\definecolor{darkgreen}{rgb}{0.1,0.7,0.1}

\newcommand{\darkgreen}[1]{\textcolor{darkgreen}{#1}}
\newcommand{\red}[1]{\textcolor{red}{#1}}
\newcommand{\thistle}[1]{\textcolor{Thistle}{#1}}
\newcommand{\apricot}[1]{\textcolor{Apricot}{#1}}
\newcommand{\melon}[1]{\textcolor{Melon}{#1}}
\newcommand{\dandelion}[1]{\textcolor{Dandelion}{#1}}
\newcommand{\green}[1]{\textcolor{OliveGreen}{#1}}
\newcommand{\lavender}[1]{\textcolor{Lavender}{#1}}
\newcommand{\mylightmagenta}[1]{\textcolor{MyLightMagenta}{#1}}
\newcommand{\blue}[1]{\textcolor{RoyalBlue}{#1}}
\newcommand{\darkorchid}[1]{\textcolor{DarkOrchid}{#1}}
\newcommand{\orchid}[1]{\textcolor{Orchid}{#1}}
\newcommand{\brickred}[1]{\textcolor{BrickRed}{#1}}
\newcommand{\peach}[1]{\textcolor{Peach}{#1}}
\newcommand{\bittersweet}[1]{\textcolor{Bittersweet}{#1}}
\newcommand{\salmon}[1]{\textcolor{Salmon}{#1}}
\newcommand{\yelloworange}[1]{\textcolor{YellowOrange}{#1}}
\newcommand{\periwinkle}[1]{\textcolor{Periwinkle}{#1}}

\newcommand{\names}[1]{\periwinkle{#1}}
\newcommand{\motcle}[1]{\mylightmagenta{#1}}
\newcommand{\classname}[1]{\darkgreen{#1}}
\newcommand{\str}[1]{\yelloworange{#1}}
\newcommand{\defun}[1]{\orchid{#1}}
\newcommand{\ti}[1]{\begin{center}\Large{\textcolor{blue}{#1}}\end{center}}
\newcommand{\alert}[1]{\thistle{#1}}
\newcommand{\lispprint}[1]{\dandelion{#1}}
\newcommand{\lispvalue}[1]{\red{#1}}
\newcommand{\tr}[1]{\texttt{\red{#1}}}
\newcommand{\emc}[1]{\red{#1}}
\newcommand{\lispobj}[1]{\green{\texttt{#1}}}
\def\prompt{{\textcolor{Orchid}{CL-USER>}}}
\newcommand{\promptp}[1]{\textcolor{Orchid}{#1>}}

\newcommand{\Comment}[1]{
\begin{center}
\textcolor{yellow}
{#1}
\end{center}
}

\def\bs{$\backslash$}
\def\inputfig#1{\input #1}
\def\inputtex#1{\input #1}

\begin{document}
\landscape
\setlength{\oddsidemargin}{1cm}
\setlength{\evensidemargin}{1cm}
\setlength{\marginparwidth}{1cm}
\setlength{\parskip}{0.5cm}
\setlength{\parindent}{0cm}
%-----------------------------------------------------------
\begin{slide}\ti{Psychology and software development}
\vskip 0.5cm
\begin{center}
Robert Strandh \\
Université de Bordeaux \\
Bordeaux, France
\end{center}
\vfill\end{slide}
%-----------------------------------------------------------
\begin{slide}\ti{Overview of talk}

\vfill\end{slide}
%-----------------------------------------------------------
\begin{slide}\ti{Two different mind sets}

According to the work of Carol Dweck, people have one of two different
\emph{mind sets}:

\begin{itemize}
\item Fixed, or
\item Growth
\end{itemize}

\vfill\end{slide}
%-----------------------------------------------------------
\begin{slide}\ti{Growth}

  \begin{itemize}
  \item Learning new things comes naturally.
  \item Failure is normal and part of the learning process.
  \item Takes pleasure in the learning process itself.
  \end{itemize}

In other words, this is your stereotypical researcher personality.

\vfill\end{slide}
%-----------------------------------------------------------
\begin{slide}\ti{Fixed}

  \begin{itemize}
  \item Needs to be perfect all the time without effort.
  \item Does not like to learn new things.
  \item Learning is seen as detrimental to performance.
  \item Failure means defective.
  \end{itemize}

\vfill\end{slide}
%-----------------------------------------------------------
\begin{slide}\ti{System 1 and system 2}

Daniel Kahneman discusses two different \emph{systems} in the brain:

\begin{itemize}
\item System 1 makes decisions very quickly without having to do any
  complicated calculations,
\item System 2 is invoked when system 1 is not able to make a
  decision, and it checks whether the decision of system 1 was
  reasonable.
\end{itemize}

You can read about it more in Kahneman's book: Thinking, fast and
slow.

\vfill\end{slide}
%-----------------------------------------------------------
\begin{slide}\ti{Questions?}

Thank you for listening!

Do you have any questions?

\vfill\end{slide}
%-----------------------------------------------------------
%% \begin{slide}\ti{}

%% \vfill\end{slide}
%-----------------------------------------------------------
%% \begin{slide}\ti{}

%% \vfill\end{slide}
%% %-----------------------------------------------------------
%% \begin{slide}\ti{}

%% \vfill\end{slide}
%--------------------------------

\end{document}
 
