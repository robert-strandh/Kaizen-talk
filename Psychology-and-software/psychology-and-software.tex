\documentclass{slides}
\usepackage[utf8]{inputenc}
\usepackage{graphics}
\usepackage{portland}
\usepackage{epsfig}
\usepackage{alltt}
\usepackage{moreverb}
\usepackage{url}
\usepackage{eurosym}
\usepackage[dvips,usenames]{color}

\definecolor{MyLightMagenta}{rgb}{1,0.7,1}
\definecolor{darkgreen}{rgb}{0.1,0.7,0.1}

\newcommand{\darkgreen}[1]{\textcolor{darkgreen}{#1}}
\newcommand{\red}[1]{\textcolor{red}{#1}}
\newcommand{\thistle}[1]{\textcolor{Thistle}{#1}}
\newcommand{\apricot}[1]{\textcolor{Apricot}{#1}}
\newcommand{\melon}[1]{\textcolor{Melon}{#1}}
\newcommand{\dandelion}[1]{\textcolor{Dandelion}{#1}}
\newcommand{\green}[1]{\textcolor{OliveGreen}{#1}}
\newcommand{\lavender}[1]{\textcolor{Lavender}{#1}}
\newcommand{\mylightmagenta}[1]{\textcolor{MyLightMagenta}{#1}}
\newcommand{\blue}[1]{\textcolor{RoyalBlue}{#1}}
\newcommand{\darkorchid}[1]{\textcolor{DarkOrchid}{#1}}
\newcommand{\orchid}[1]{\textcolor{Orchid}{#1}}
\newcommand{\brickred}[1]{\textcolor{BrickRed}{#1}}
\newcommand{\peach}[1]{\textcolor{Peach}{#1}}
\newcommand{\bittersweet}[1]{\textcolor{Bittersweet}{#1}}
\newcommand{\salmon}[1]{\textcolor{Salmon}{#1}}
\newcommand{\yelloworange}[1]{\textcolor{YellowOrange}{#1}}
\newcommand{\periwinkle}[1]{\textcolor{Periwinkle}{#1}}

\newcommand{\names}[1]{\periwinkle{#1}}
\newcommand{\motcle}[1]{\mylightmagenta{#1}}
\newcommand{\classname}[1]{\darkgreen{#1}}
\newcommand{\str}[1]{\yelloworange{#1}}
\newcommand{\defun}[1]{\orchid{#1}}
\newcommand{\ti}[1]{\begin{center}\Large{\textcolor{blue}{#1}}\end{center}}
\newcommand{\alert}[1]{\thistle{#1}}
\newcommand{\lispprint}[1]{\dandelion{#1}}
\newcommand{\lispvalue}[1]{\red{#1}}
\newcommand{\tr}[1]{\texttt{\red{#1}}}
\newcommand{\emc}[1]{\red{#1}}
\newcommand{\lispobj}[1]{\green{\texttt{#1}}}
\def\prompt{{\textcolor{Orchid}{CL-USER>}}}
\newcommand{\promptp}[1]{\textcolor{Orchid}{#1>}}

\newcommand{\Comment}[1]{
\begin{center}
\textcolor{yellow}
{#1}
\end{center}
}

\def\bs{$\backslash$}
\def\inputfig#1{\input #1}
\def\inputtex#1{\input #1}

\begin{document}
\landscape
\setlength{\oddsidemargin}{1cm}
\setlength{\evensidemargin}{1cm}
\setlength{\marginparwidth}{1cm}
\setlength{\parskip}{0.5cm}
\setlength{\parindent}{0cm}
%-----------------------------------------------------------
\begin{slide}\ti{Psychology and software development}
\vskip 0.5cm
\begin{center}
Robert Strandh \\
Université de Bordeaux \\
Bordeaux, France
\end{center}
\vfill\end{slide}
%-----------------------------------------------------------
\begin{slide}\ti{Overview of talk}

We investigate the effect on software development by the following
phenomena:

  \begin{itemize}
  \item The \emph{mind set} as investigated by Carol Dweck.
  \item The \emph{dual process theory} suggested by William James, and
    made popular by Daniel Kahneman.
  \item The \emph{planning fallacy} phenomenon investigated by
    Daniel Kahneman.
  \end{itemize}

\vfill\end{slide}
%-----------------------------------------------------------
\begin{slide}\ti{Two different mind sets}

According to the work of Carol Dweck, people have one of two different
\emph{mind sets}:

\begin{itemize}
\item Fixed, or
\item Growth
\end{itemize}

\vfill\end{slide}
%-----------------------------------------------------------
\begin{slide}\ti{Growth}

  \begin{itemize}
  \item Learning new things comes naturally.
  \item Failure is normal and part of the learning process.
  \item Takes pleasure in the learning process itself.
  \end{itemize}

In other words, this is your stereotypical researcher personality.

\vfill\end{slide}
%-----------------------------------------------------------
\begin{slide}\ti{Fixed}

  \begin{itemize}
  \item Needs to be perfect all the time without effort.
  \item Does not like to learn new things.
  \item Learning is seen as detrimental to performance.
  \item Failure means defective.
  \end{itemize}

\vfill\end{slide}
%-----------------------------------------------------------
\begin{slide}\ti{Mind set and software development}

In a team, a person with a fixed mind set is often well liked,
even admired.

He or she is fast and competent in getting things done.

\vfill\end{slide}
%-----------------------------------------------------------
\begin{slide}\ti{Mind set and software development}

But when things might change (new tools, new methods, new programming
language), the person with a fixed mind set feels threatened.

Because, if things change, that person may have to learn something new
which he or she is not comfortable with.

\vfill\end{slide}
%-----------------------------------------------------------
\begin{slide}\ti{Mind set and software development}

As a result, instead of accepting the changes, the person with fixed
mind set tries to make sure the changes do not happen.

The person will not hesitate using methods such as lying about the
nature of the changes, or ridiculing the people in favor of the
changes.

Since the person is often admired, these methods will frequently be
effective.

\vfill\end{slide}
%-----------------------------------------------------------
\begin{slide}\ti{Example of different mind sets}

Let's say John and Mary are part of a development team.  John has a
fixed mind set, whereas Mary has a growth mind set.

Mary might suggest that the programming language Haskell be used for
the next project, because it seems like a good fit.

\vfill\end{slide}
%-----------------------------------------------------------
\begin{slide}\ti{Example of different mind sets}

John might say:
\begin{itemize}
\item that Haskell is slow,
\item that automatic memory management is not adapted to this project,
\item that Haskell will take too long to learn which will increase the
  cost of the project,
\item that even the people that created Haskell are now abandoning it,
  and even something like
\item that Haskell is only for sissies.
\end{itemize}

\vfill\end{slide}
%-----------------------------------------------------------
\begin{slide}\ti{Anecdote about mind set}

I once worked for what is now called ABB.

The company had a policy of using mainframes only from IBM.

I worked for the divisions for control systems, and we wanted
something else (Multics, the precursor of Unix).

One person working for the research department work extremely hard to
make sure that we would not buy Multics, because he had invested all
his knowledge in IBM VM/CMS.

It all ended well.  We managed to convince the company despite this
person.

\vfill\end{slide}
%-----------------------------------------------------------
\begin{slide}\ti{System 1 and system 2}

Daniel Kahneman discusses two different \emph{systems} in the brain:

\begin{itemize}
\item System 1 makes decisions very quickly without having to do any
  complicated calculations,
\item System 2 is invoked when system 1 is not able to make a
  decision, and it checks whether the decision of system 1 was
  reasonable.
\end{itemize}

You can read about it more in Kahneman's book: Thinking, fast and
slow.

\vfill\end{slide}
%-----------------------------------------------------------
\begin{slide}\ti{System 1 and system 2 example}

On example in Kahneman's book is this:

Two items A and B together cost 1\euro{}.  Item B costs 0.9\euro{}
more than item A.  How much does item A cost?

\vfill\end{slide}
%-----------------------------------------------------------
\begin{slide}\ti{System 1 and Kaizen}

Example:  You have the following code:

\begin{verbatim}
int
fun(int x)
{
  int y = fun2(x, 3);
  return y * x;
}
\end{verbatim}

And your cursor is positioned on the \texttt{return} statement.  You
want to go to the beginning of the definition of the function
\texttt{fun}.

\vfill\end{slide}
%-----------------------------------------------------------
\begin{slide}\ti{System 1 and Kaizen}

You have a choice between:

\begin{itemize}
\item Looking in the manual for your text editor in order to find the
  command that will go to the beginning of a definition.
\item Using arrow keys (or equivalent) or the mouse to go there.
\end{itemize}

System 1 will always tell you to use the second option, even though
the first option will pay off within a few days.

This topic is treated in depth in my talk ``Kaizen for everyone''.

\vfill\end{slide}
%-----------------------------------------------------------
\begin{slide}\ti{System 1 and strategic decisions}

The following chain of reasoning:

\begin{quotation}
For the next project, let us choose Java as the programming language.
It is the best choice because everyone in the team already knows
Java.
\end{quotation}

is a result of system 1 making a decision.  As I argue in my talk
``The influence of the programming language on productivity'', the
cost of learning a new language might be very small compared to the
gain in increased productivity.
\vfill\end{slide}
%-----------------------------------------------------------
\begin{slide}\ti{Fighting system 1}

Force system 2 into action.

For each alternative:

\begin{itemize}
\item Make a complete analysis of the costs and benefits.
\item Make a risk analysis with likelihoods and associated costs.
\end{itemize}

Decide on one of the alternatives based on these number and on your
intuition for everything that can not be precisely computed.

\vfill\end{slide}
%-----------------------------------------------------------
\begin{slide}\ti{Planning fallacy}

People tend to underestimate the time it takes to complete a task.

Example from Buehler, Griffin, and Ross: Exploring the "planning
fallacy": Why people underestimate their task completion times.

More than 30 psychology students were asked to estimate how long it
would take to finish their senior theses.

\begin{itemize}
\item Best scenario estimate: less than 28 days.
\item Average estimate: around 40 days.
\item Worst scenario estimate: almost 50 days.
\end{itemize}

Result: Average time for finishing: more than 55 days.

\vfill\end{slide}
%-----------------------------------------------------------
\begin{slide}\ti{Overconfidence and software}

In software development, overconfidence results in impossible
deadlines for projects.

This special case of overconfidence is called the \emph{planning
  fallacy}. 

\vfill\end{slide}
%-----------------------------------------------------------
\begin{slide}\ti{Fighting overconfidence}

A complete \emph{risk analysis} will expose factors that may delay the
project.

\vfill\end{slide}
%-----------------------------------------------------------
\begin{slide}\ti{Possible future talks}

{\small
  \begin{itemize}
  \item Kaizen for everyone
  \item My ideal operating system
  \item Influence of the programming language on productivity
  \item Choosing a programming language
  \item Algorithms and data structures
  \item Object-oriented programming with generic functions
  \item Parsing technology
  \item Compiler technology (optimization)
  \item Operating-system techniques
  \item Memory-management techniques
  \end{itemize}}

\vfill\end{slide}
%-----------------------------------------------------------
\begin{slide}\ti{Questions?}

Thank you for listening!

Do you have any questions?

\vfill\end{slide}
%-----------------------------------------------------------
%% \begin{slide}\ti{}

%% \vfill\end{slide}
%-----------------------------------------------------------
%% \begin{slide}\ti{}

%% \vfill\end{slide}
%% %-----------------------------------------------------------
%% \begin{slide}\ti{}

%% \vfill\end{slide}
%--------------------------------

\end{document}
 
